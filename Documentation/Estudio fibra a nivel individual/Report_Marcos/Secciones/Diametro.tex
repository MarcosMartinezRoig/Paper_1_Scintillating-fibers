Procedemos a comprobar el diámetro de las mismas cuyo valor proporcionado por Saint-Gobain es de $1~\mm$. Para ello medimos el diámetro de 3 fibras diferentes de cada tipo con ayuda de un microscopio electrónico del ICMOL, cuya incertidumbre se encuentra entorno al nanómetro. Los resultados promediado se muestra en la tabla \ref{diametros} .

\begin{table}[H]
\begin{center}
\begin{tabular}{l | c | c | c | c }
Tipo de fibra & Electronical microscope $(\sigma_{el}\approx 10^{-6} mm)$\\
\hline \hline
No Clad (mm)& $1.00 \pm 0$\\ 
Single clad (mm) & $1.00 \pm 0$\\
Multi clad (mm)  & $1.00 \pm 0$\\
\end{tabular}
\caption{Diámetro promediado sobre tres muestras\label{diametros}}
\end{center}
\end{table}

Podemos apreciar que el diámetro de todas las fibras que se midieron fue exactamente un milímetro con una resolución de $10^{-6}~\mm$ ofrecida por el microscopio electrónico ya que la desviación estandar del promedio es exactamente cero. 

Con este estudio se ha comprobado que la adición de clad implica la retirada de núcleo centelleador, es decir, Saint-Gobain ofrece una mejora en la colección de luz a costa de un menor núcleo centelleador. 

Hay que tener en cuenta que esta retirada de núcleo centelleador no afecta al proyecto Tritium ya que los electrones procedentes de la desintegración del tritio, y que aportarán señal de forma apreciable, únicamente recorrerán un máximo de 5 o 6$~\mu\meter$ y, incluso con los multiclads, la fibra posee un núcleo de mas de $900~\mu\meter$ de diámetro. El problema aparece únicamente con los clads comerciales ya que estos poseen un espesor mínimo de $40~\mu\meter$, por lo que, si tenemos en cuenta el recorrido libre medio de estos eventos, podemos ver que ninguno conseguirá llegar al núcleo de la fibra y aportar señal al detector. Se esta estudiando la posibilidad de crear núcleos que permitan la recolección total de los fotones pero lo suficientemente finos para que no afecten de forma importante a la tasa de eventos de tritio que consigan llegar al nucleo.