En primer lugar, antes de empezar con el estudio de la caracterización de las fibras, necesitamos realizar una serie de labores de preparación del dispositivo experimental para asegurar el correcto funcionamiento y entendimiento del mismo. Para ello realizaremos tres comprobaciones:
\begin{itemize}
\item{} En primer lugar, realizaremos una comprobación para asegurar que el diámetro de las fibras se corresponde con el proporcionado por la empresa Saint-Gobain.

\item{} En segundo lugar realizaremos una estudio para verificar la hermiticidad a la luz de la caja oscura en el interior de la cual se desarrollará el estudio.

\item{} En último lugar realizaremos un estudio del PMT utilizado en el experimento para asegurar la linealidad de su respuesta del mismo en el intervalo de interes en el experimento y, en concreto, en el detector Tritium ya que, aunque su iniciativa primera será utilizar SiPM también se estudiará la posibilidad de emplear PMTs.
\end{itemize}