Finally I are going to explain the conclusions which we have obtained in these studies and next step which we will do for following these one.
\begin{itemize}
\item{} First of all we have prepared a new prototype which is apparently similar to the Valencia prototype but it has several important differences from point of view of the physics. Beforehand we can't know what is the best configuration because we can't simulate these prototypes. The reason of this is that there are a lot of important concepts, like the imperfections of the fiber shape, which will affect at the signal of every prototype differently. The problem is that, nowadays, there's not exist any simulator which can do this type of task.

As a result, the only way, which we have for checking what is the best configuration for the final prototype, is to put both prototypes in Almaraz power nuclear plant and we start to mesure with both at the same time. When we analyze the results of every prototype we will can discuss what is the best configuration for this task. 

As I said before, we have to take into account that the internal volum is different in every prototype so the quantity of tritium water, which is the radiactive source that we pretend to mesure, will be different in every prototype. Therefore, in order that we are able to compare the results between both, we have to scale the results of each one at the same internal volum.

\item{} Second, by the moment, the electronics, which control this signal, is the typical electronics for general use because this is the electronics which we have in our laboratories and it is the electronics which we know. In this way we reduce the uncertainty in our study.

When we calibrate our prototype we will have to develop a new type of electronics which will have very low noise and we will have to study this electronics for undestanding it before we install all this thing in Almaraz power nuclear plant. It will be really important step because our main problem is that the tritium signal, with which we pretend to work, will be very low so we need that the noise in our experiment is really low so that we can obtain a good tritium signal in less than ten minuts.

\item{} Third we have to take into account that we have to polish the fibers which we have prepared for the Aveiro prototype. This work was easier for Valencia prototype because this has only $64$ fibers but the Aveiro prototype has $350$ fibers. As I have explained in the section \ref{sec:polishing} we need too much time for doing this task handly so, as there are not any device which do this task, we have start to develop this kind of machine.

Until now we have checked that, with arduino technology, we can program this machine for describing the way which we want at the speed which we want. We are able to do several figures with several speeds.

The main problem which we have found in our first prototype is that the motors, which we use for moving this machine, have some limitations like the torque. This torque imposes a limit in the maximum speed with which we can move this piece and, therefore, it imposes a limit in the minimum time which you need for polishing the fibers. Although we have get a speed with which we can do this task with an acceptable time we want to improve the system for increasing this limit in the speed. For doing this task we have several options like use a fast screws or to use other type of motors with bigger torque. In this case we will can check what is the minimum time which we need for polishing the fibers without we damage the fibers.

\item{} Finally, we have to take into account that there's not exist any commercial devices like this one. Therefore, if you have to take into account that, nowadays, there are a lot of experiments, which need to use a big quantity of fibers in their configuration, you can sense that the device, which we are developing, is really useful. 

As a result, if this devices work and we create a easy display so that users can manipulate the parameters of the moviment, which is described by the device, we could patent this one.
\end{itemize}


